\documentclass[11pt,twosided]{article}

\usepackage{9month}
\title{Properties of the UEG}
\begin{document}
\maketitle
\section{Free Electrons}
N free electrons moving in a cubic box of length $L = L_x = L_y = L_z$ with periodic boundary conditions and $\mathbf{k} = 2\pi/L(n_x,n_y,n_z)$.
\begin{align}
    \hat{H} &= \sum_{\mathbf{k}} \varepsilon_{\mathbf{k}} c_{\mathbf{k}}^{\dagger}c_{\mathbf{k}}\\
    \varepsilon_{\mathbf{k}} &= \frac{\hbar^2 \mathbf{k}^2}{2 m_e}\label{free_elec}\\
    \mathbf{k}^2 &= \frac{2 \pi }{L^2} (n_x^2 + n_y^2 + n_z^2)
\end{align}
So,
\begin{align}
    \varepsilon_{\mathbf{k}} &= \frac{2 \pi^2 \hbar^2 \mathbf{n}^2}{ m_e L^2}\\
                             & = \frac{\hbar^2}{m_e a_0^2} \frac{2 \mathbf{n}^2}{\tilde{L}}\\
                             & = 1 E_h  \frac{2 \mathbf{n}^2}{\tilde{L}}
\end{align}
where,
\begin{align}
    1 E_h &= \frac{\hbar^2}{m_e a_0^2}\\
    a_0 & = \mathrm{bohr \ radius}\\
    \tilde{L} &= L/a_0.
\end{align}
Our system is then defined by $N_e$, $\tilde{L}$ and $r_s$, where $r_s$ is found from
\begin{align}
    \frac{4}{3}\pi r_s^3 &= \frac{L^3}{N_e}\\
    r_s &= \tilde{L} a_0 \Big(\frac{3}{4\pi N_e}\Big)^{1/3}
\end{align}
The input parameters to HANDE are then $\tilde{r}_s = r_s/a_0$ and $N_e$ which define the system and energies are measured in Hartrees.
\section{Zero temperature properties}
\subsection{Fermi vector/energy}
For a large box the fermi vector is found by equating the volume of a sphere in $k$-space to that found by filling the space by boxes of side $2\pi/L$ corresponding volume of $k$-space which contains one $k$ point.
The energy is found using \Cref{free_elec}.
Explicitly:
\begin{align}
    4/3 \pi k_F^3 &= n_s N_e\frac{(2\pi)^3}{V}\\
            k_F &= (6 \pi^2 n_s n)^{1/3}\\
            E_F &= \frac{1}{2}(6 \pi^2 n_s n)^{2/3}
\end{align}
where $n_s$ is a spin degeneracy factor (= 1 for spin polarised case), and $n = N/V$.
For a finite system or small number of electrons the above approximation in equating the volume of a sphere to that of a number of boxes won't be good so we'll use the usual definition of the Fermi energy as being the energy of the first unoccupied single particle energy level \footnote{Apparently people also use the energy of the last occupied energy level but we'll stick with the above definition}.
\subsection{Total Energy}
The total energy is found by occupying all the single particle energy levels with the appropriate number of up and down spin electrons.
\begin{align}
    E_{\mathrm{total}} = \sum_{\mathbf{k}_{\mathrm{occ}}} \varepsilon_\mathbf{k}
\end{align}
Continuum:
\begin{align}\label{tot_e}
    E_{\mathrm{total}} &= \bra{\mathrm{FS}}\sum_{\mathbf{k}} \varepsilon_{\mathbf{k}} \hat{n}_\mathbf{k} \ket{\mathrm{FS}}\\
                       &= \sum_{\mathbf{k}}\varepsilon_{\mathbf{k}} \ \theta(\mathbf{k} - \mathbf{k}_f)\\
                       &\stackrel{V\rightarrow\infty}{=} \frac{V}{2\pi^3}\int d\mathbf{k} \ \varepsilon_{\mathbf{k}} \ \theta(\mathbf{k} - \mathbf{k}_f)\\
                       &=\frac{1}{2} \frac{V}{2\pi^3} \int d\mathbf{k} \ k^2 \ \theta(\mathbf{k} - \mathbf{k}_f)\\
                       &=\frac{4\pi}{2} \frac{V}{2\pi^3} \int_0^{k_F} dk \ k^4  \\
                       &=\frac{4\pi}{2}\frac{(6 \pi^2 n_s N_e)}{V}\frac{V}{2\pi^3}\frac{k_F^2}{5}\\
                       &\stackrel{\sigma}{=}\sum_\sigma\frac{3}{5}n_sE_F\\
                       &=\frac{3}{5}E_F
\end{align}
\section{Finite Temperature}
The density matrix for a system with $N$ particles is
\begin{align}\label{canonical}
    Z_C &= e^{-\beta\hat{H}}\\
        &= \sum_i e^{-\beta E_i},
\end{align}
where the $E_i$ are many particle eigenstates.
For the free electron gas these can be calculated by enumerating all possible combinations of filling the single particle eigenvalues with $N$ electrons, but this is difficult to do in practice.
It is easier to work with the grand canoncical partition function:
\begin{align}\label{grand_can}
    Z_G &= e^{-\beta(\hat{H}-\mu\hat{N})}\\
        &= \sum_N^\infty\sum_{\substack{\{n_\mathbf{k}\}\\ \sum n_{\mathbf{k}}=N}} e^{-\beta\sum_{\mathbf{k}}(\varepsilon_\mathbf{k}-\mu)n_{\mathbf{k}}}
\end{align}
Here the second summation is a restricted sum over the set of all possible occupation numbers, $\{n_{\mathbf{k}}\}$, such that the total number of partiles is $N$.
We could equally well write \Cref{canonical} as
\begin{equation}
    \sum_{\substack{\{n_\mathbf{k}\}\\ \sum n_{\mathbf{k}}=N}} e^{-\beta\sum_{\mathbf{k}}\varepsilon_\mathbf{k}n_{\mathbf{k}}},
\end{equation}
but this is only true for non-interactings systems where the single particles excitations, $\varepsilon_\mathbf{k}$, are well defined (or known in the case of quasi-particles?).

Anyway, the use of the grand canonical partition function simplifies matters as we no longer need to worry about the enumeration problem and can just deal with the single particle dispersion.
This can be seen by noting that
\begin{align}
    Z_G &= \sum_{n_0}\sum_{n_1}\dots (e^{-\beta(\varepsilon_0-\mu)})^{n_0}(e^{-\beta(\varepsilon_1-\mu)})^{n_1} \dots \\
        &= \Big[\sum_{n_0}(e^{-\beta(\varepsilon_0-\mu)})^{n_0}\Big]\Big[\sum_{n_1}(e^{-\beta(\varepsilon_1-\mu)})^{n_1}\Big]\dots \\
        &= \prod_\mathbf{k}\Big[\sum_n (e^{-\beta(\varepsilon_\mathbf{k}-\mu)})^n\Big] \\
        &= \prod_\mathbf{k}(1+e^{-\beta(\varepsilon_\mathbf{k}-\mu)}).
\end{align}
The mean total number of particles, $\bar{N} = \sum_{\mathbf{k}}\langle \hat{n}_{\mathbf{k}}\rangle$, can be calculated as
\begin{align}\label{nav}
    \bar{N} &= z\frac{\partial}{\partial z} \log Z_G \\
            &= \sum_{\mathbf{k}}\frac{z e^{-\beta\varepsilon_{\mathbf{k}}}}{1+z e^{-\beta\varepsilon_{\mathbf{k}}}}\\
            &= \sum_{\mathbf{k}}\frac{1}{ze^{ \beta\varepsilon_k} + 1},\label{av_p}
\end{align}
where $z=e^{\beta\mu}$ is the fugacity, which I think is soley introduced so that those ugly brackets aren't needed in the exponentials, and it makes derivatives of quantities with respect to it nicer.
This is the usual Fermi-factor
\begin{equation}\label{fermi}
    f(\varepsilon_k) = \frac{1}{e^{\beta(\varepsilon_k-\mu)} + 1}.
\end{equation}
which describes how the single-particle levels are occupied at finite temperature and reduces to the usual step function appearing in \Cref{tot_e} as $\beta\rightarrow\infty$.

Observables can now be calculated in an ensemble with a fixed particle number by inverting \Cref{av_p} to find the appropriate $\mu$ which results in the desired $N$.
This amounts to finding the roots of $N_e - \sum_\mathbf{k} \frac{1}{e^{\beta(\varepsilon_\mathbf{k} - \mu)}+1}$, or more explicitly, the value of $\mu$ when $N_e - F(\mu) = 0$, which can be done using Newton-Raphson or whatever.

\section{Sums and Integrals}

A lot of the things we want to calculate for the free electron gas and Hartree-Fock approximation for the UEG use the approximation of replacing sums over $\mathbf{k}$ with integrals.
This is justified in the large volume limit, but for the small systems we'll likely be restricted to, perhaps this isn't a very good approximation.
In this section I'll list some quantites which we might want to check carefully when comparing results with other people's work.
But first the approximation.

We use the usual definition of a Riemann sum to interchange sums and integrals, so

\begin{equation}
    \lim_{\Delta x \rightarrow 0} \sum_i f(x_i) \Delta x_i = \int f(x) dx.
\end{equation}

For example, the total energy at $T=0$ is given as
\begin{align}
    E &= \sum_{\mathbf{k}} \varepsilon_{\mathbf{k}} \theta(\mathbf{k}-\mathbf{k_f})\\
      &\approx \frac{V}{(2\pi)^3} \int d\mathbf{k} \ \varepsilon_{\mathbf{k}} \ \theta(\mathbf{k}-\mathbf{k_f}),
\end{align}
using the fact that $\Delta k_\alpha = 2\pi/L_\alpha$.

When $L$ is small or, in our case, when we desire a high density with a small number of electrons, this approximation is not likely to be a good one.
This might be annoying when comparing to previous results for the correlation energy.

Before I list some things I will introduce the Fermi integral, which is often used in discussions of the UEG at finite temperatures, and is how Clark \emph{et al.} calculate the correlation energy found in the supplementary material of their PRL.

To begin consider \Cref{av_p}, assuming large $V$ throughout and reintroducing units,
\begin{align}
    \langle \hat{N} \rangle &= \sum_{\mathbf{k}}\frac{1}{ze^{ \beta\varepsilon_k} + 1}\\
                            &= \frac{V}{(2\pi)^3}\int d \mathbf{k} \ \frac{1}{ze^{ \beta\varepsilon_k} + 1}\\
                            &= \frac{V}{(2\pi)^3}4\pi\int_0^{\infty} dk \ \frac{k^2}{ze^{ \beta\varepsilon_k} + 1}\\
                            &= \frac{V}{(2\pi)^3}4\pi\Big(\frac{2m}{\hbar^2}\Big)^{3/2}\int_0^\infty d\varepsilon  \ \frac{\varepsilon^{1/2}}{ze^{ \beta\varepsilon} + 1}\\
                            &= \frac{V}{(2\pi)^3}4\pi\Big(\frac{2m}{\hbar^2}\Big)^{3/2}\beta^{-3/2}\int_0^\infty dx\ \frac{x^{1/2}}{e^{ \beta(x-\eta)} + 1}\\
                            &= \frac{V}{(2\pi)^3}4\pi\Big(\frac{2m}{\hbar^2}\Big)^{3/2}\beta^{-3/2} I(1/2,\eta)
\end{align}
where,
\begin{equation}
    I(\nu,\eta) := \int_0^\infty dx\ \frac{x^{\nu}}{e^{ \beta(x-\eta)} + 1}
\end{equation}
is the Fermi integral and $\eta = \mu/kT$.

This integral pops up a lot for instance
\begin{equation}
    \frac{2}{3}\Theta^{-3/2} = I(1/2,\eta)
\end{equation}
which can be seen by noting that $E = \frac{\hbar^2}{2m} (3 \pi^2 n)^{2/3} = kT$, and
\begin{equation}
    E = \frac{V}{2\pi^2}\Big(\frac{2m}{\hbar^2}\Big)^{3/2} \beta^{-5/2} I(3/2,\eta)
\end{equation}
which I think differs from what Clark has in the supplementary material, where the units aren't correct.
\section{Approximate Theories}
In analogy with zero temperature Hartree-Fock theory one can define what is sometimes called the `Hartree-Fock exchange'\footnote{This is somewhat confusingly labelled $E_{HF},x$ in Brown \emph{et al.}, here we'll call it $\Omega_1$ which is more in keeping with the literature around this subject. Although this is also potentially confusing as $\Omega$ is usually reserved for the grand potential...} energy as
\begin{align}
\Omega_1 &= \langle \sum_{\vb{k}} \sum_{\vb{q}} \hat{n}_k \hat{n}_q \rangle\\
	  &=  \sum_{\vb{k}} \sum_{\vb{q}} v(k-q) f_k f_q\\
	  &= \frac{V}{(2\pi)^6}\int\int d\vb{k}d\vb{q} v(k-q) f_k f_q.
\end{align}
It should be stressed that this is the first order contribution found from finite-temperature perturbation theory and not the result found from finite temperature Hartree-Fock theory, which will be discussed in the next section.
$E_x^0$ can be simplified to yield
\begin{equation}
\Omega_1 = -\frac{V}{2\pi^3\beta^2}\int_{-\infty}^{\eta} d\xi \ I(-1/2, \xi)^2
\end{equation}
To see this we first start from the expression for $\omega_1$ in real space
\begin{align}
\Omega_1 = - \int \int dr dr' \frac{1}{|r-r'|} |G(r-r')|^2,
\end{align}
where
\begin{equation}
G(x) = \frac{1}{2\pi^2x}\int_0^{\infty} dp \ f_p p \sin px.
\end{equation}
The idea is to get an expression as in integral over $z$, so differentiating .. with respect to z we find
\begin{equation}
\frac{d\Omega_1}{dz} = - 2\int \int dr dr' v(r-r') G(r-r')\frac{dG}{dz},
\end{equation}
where
\begin{equation}
\frac{dG}{dz} = \frac{1}{2\pi^2x}\int_0^{\infty} \frac{\partial f_p}{\partial z} \ p \sin px.
\end{equation}
We can further simplify this by noticing that
\begin{align}
\frac{\partial f_p}{\partial z} &= \frac{z^{-2} e^{\beta p^2/2}}{(z^{-2} e^{\beta p^2/2}+1)^2}\\
							   &= \frac{1}{ \beta zp}\frac{z^{-1} \beta p e^{\beta p^2/2}}{(z^{-2} e^{\beta p^2/2}+1)^2}\\
							   &= \frac{1}{\beta zp}\frac{\partial f_p}{\partial p},
\end{align}
so that
\begin{align}
\frac{dG}{dz} &= \frac{1}{2\pi^2xz}\int_0^{\infty} d p \ \frac{\partial f_p}{\partial p} \sin px \\
			  &= \frac{1}{2\pi^2z}\int_0^{\infty} d p f_p \cos px,
\end{align}
where the total derivative term vanishes because $\sin (0) = n_p(\infty) = 0$.
If we now define
\begin{align}
g(x) = \int_0^{\infty} d p f_p \cos px,
\end{align}
we notice that
\begin{align}
\frac{dg(x)}{dx} &= -\int_0^{\infty} d p f_p p \sin px\\
				&= -2\pi^2 x G(x).
\end{align}
So that
\begin{align}
\frac{d\Omega_1}{dz} = \frac{2}{(2\pi^2)^2}\int \int dr dr' \frac{1}{|r-r'|^2} g(|r-r'|) \frac{dg(|r-r'|)}{d|r-r'|}.
\end{align}
Noticing that everything is now a function of $|r-r'|$ we can carry out one of the $r$ integrals to get a volume factor and rewrite everything in terms of $x = |r-r'|$, so
\begin{align}
\frac{d\Omega_1}{dz} &= \frac{8\pi V}{(2\pi^2)^2z} \int_0^{\infty} d x g(x) \frac{dg(x)}{dx}\\
					 &= \frac{-V}{\pi^3z\beta} g(0)^2.
\end{align}
That the upper limit in the integration by parts vanishes can be seen by looking at
\begin{align}
g(x) &= \int_0^{\infty} dp f_p \sin px\\
								&= \int_0^{\infty} dp f_p \frac{1}{x}\frac{d}{dx}\cos px\\
								&= f_p \frac{\sin px}{x}\big|_0^{\infty} - \frac{1}{x}\int_0^\infty d p \frac{\partial f_p}{d p} \sin px
\end{align}
and it is then obvious that $\lim_{x\rightarrow\infty} g(x) = 0$.
Integrating equation (), we finally find that
\begin{align}
\Omega_1 &= -\frac{V}{\pi^3\beta}\int_0^{z} d\xi g(0)^2\\
 		 &= -\frac{V}{2\pi^3\beta^2}\int_0^{z} \Big(\int_0^{\infty}dx\frac{x^{-1/2} }{z^{-1} e^x + 1}\Big)^2\\
 		 &= -\frac{V}{2\pi^3\beta^2}\int_{-\infty}^{\eta} d\xi \ I(-1/2, \xi)^2,
\end{align}
as claimed (this has a plus sign in perrot-dharma).
\section{Thermal Hartree-Fock}
The grand canonical density matrix, $\rho$, is found by minimising the Helmholtz free energy
\begin{align}
F &= E - \mu N + TS,\\
  &= \tr{\hat{\rho}(\hat{H} - \mu \hat{N})} + \beta^{-1}\tr{\hat{\rho}\log \hat{\rho}}
\end{align}
with respect to variations in $\rho$.
Carrying this out, one finds that
\begin{align}
\rho &= \frac{1}{Z}e^{-\beta(\hat{H}-\mu \hat{N})},\\
Z &= \tr{e^{-\beta(\hat{H}-\mu \hat{N})}}.
\end{align}
The thermal Hartree-Fock approximation amounts to finding the trial density matrix, $\hat{\rho}_T$, of the form
\begin{align}
\hat{\rho}_T &= \frac{1}{Z_T} e^{-\beta \hat{H}_T}\\
			 &= \frac{1}{Z_T} e^{-\beta \sum_{ij} \gamma_{ij} \hat{a}^{\dagger}_i \hat{a}_j},
\end{align}
which minimises $F$.
Working in the basis in which $H_T$ is diagonal we can write
\begin{align}
Z_T &= \prod_i (1+e^{-\beta \gamma_{ii}}),\\
\hat{\rho_T} &= \prod_i(\hat{n}_if_i + (1-f_i)(1-\hat{n}_i))),
\end{align}
and by noticing that $e^{-\beta \gamma_{ii}\hat{n}_i} = 1 + (e^{-\beta \gamma_{ii}}-1)\hat{n}_i$ we can find that
\begin{align}
\hat{\rho} &= \prod_i\frac{1+\hat{n}_i (e^{-\beta \gamma_{ii}}-1)}{1+e^{-\beta \gamma_{ii}}}\\
		   &= \prod_i  (\hat{n}_if_i + e^{\beta\gamma_{ii}}f_i(1-\hat{n}_i) )\\
		   &= \prod_i ( \hat{n}_i f_i + (1-f_i)(1-\hat{n}_i)).
\end{align}
To evaluate the traces of $\rho_T$ with $\hat{H}$ we need to know
\begin{align}
\langle \ac{i} \ad{j} \rangle_0 = f_{ij} \delta_{ij},
\end{align}
and
\begin{align}
\langle \ac{i} \ac{j} \ad{k} \ad{l} \rangle_0 = \langle \ac{i} \ad{l} \rangle_0 \langle\ac{j}  \ad{k} \rangle_0 - \langle \ac{i} \ad{k} \rangle_0 \langle\ac{j} \ad{l} \rangle_0,
\end{align}
by Wick's theorem, so that
\begin{align}
F &= \sum_i t_{ii} f_i + \sum_{ij} v_{ij} f_i f_j + 1/\beta \sum_i f_i \log f_i + (1-f_i) \log (1-f_i),\\
  &= \tr{tf + v f\times f)} + \tr{f \log f + (1-f) \log (1-f)}
\end{align}
where the traces have been evaluated in the basis which diagonalises $H_T$ and $v_{ij} = \bra{ij} v \ket{ij} - \bra{ij} v \ket{ji}$.
Varying $F$ we find
\begin{align}
\delta F &= \tr{\delta f (t + v f)} + 1/\beta \tr{\delta f \log (\frac{f}{1-f})}
\end{align}
which is minimal when
\begin{align}
\beta \varepsilon_{ij} = - \log (\frac{f}{1-f})
\end{align}
or that
\begin{align}
\gamma_{ii} = t_{ii} + \sum_{ij} v_{ij} f_j
\end{align}
with the self consistency condition
\begin{align}
f_i = \frac{1}{e^{\beta \gamma_{ii}} + 1}.
\end{align}
\section{DMQMC}
\end{document}
